\bccTitle{Meet in the City}

\bccBox{Story}{

Two beavers are planning on meeting somewhere in their city. 
Their current locations are shown on the following map of the city along with the locations of 
water \raisebox{-1mm}{\includegraphics[width=.03\textwidth]{images/2021-LT-01-icon-water.png}}, 
two bikes \raisebox{-1mm}{\includegraphics[width=.043\textwidth]{images/2021-LT-01-icon-bike.png}},
and two cars \raisebox{-1mm}{\includegraphics[width=.045\textwidth]{images/2021-LT-01-icon-car.png}}.

\begin{center}
\includegraphics[scale=0.5]{images/2021-LT-01-map-01.png}
\end{center}

The beavers can only move from one square on the map to another square that is horizontally or vertically adjacent to their square and does not contain water.\\

They can move 1 square in 1 minute when walking. However, if they reach a square with a bike or a car, then they can use it to travel faster. They can move 1 square in 30 seconds while on a bike, and they can move 1 square in 12 seconds while in a car.

}

\bccBox{Question}{

What is the least amount of time needed for the beavers to meet on the same square together?

\begin{enumerate}[(A)]
%\item 3 minutes
%\item 4 minutes
%\item 5 minutes
%\item 6 minutes
\item 3 minutes and 48 seconds
\item 4 minutes
\item 4 minutes and 12 seconds
\item 5 minutes
\end{enumerate}

}

\begin{solution}

\newpage

\bccBox{Answer}{

(B) 4 minutes

}

\bccBox{Explanation of Answer}{

If we convert all the time to minutes, we see that in 1 minute a beaver can move 1 square when walking, or 2 squares while on a bike, or 5 squares while in a car.\\

\begin{minipage}{0.43\textwidth}
\includegraphics[width=0.74\textwidth]{images/2021-LT-01-map-02.png}
\end{minipage}\hspace{-15mm}
\begin{minipage}{0.57\textwidth}

The beavers can arrive at the same square after 4 minutes if they take the routes as shown. Each individual arrow represents 1 minute of travel time.\\

To see why the beavers cannot arrive at the same square after fewer than 4 minutes, we determine all squares that one beaver could reach in 3~minutes and compare this to the squares that the other beaver could reach in 3~minutes.
\end{minipage}

\vspace{5mm}

\begin{minipage}{0.57\textwidth}
The three diagrams given show the squares that the beavers can reach after 1 minute, 2 minutes, and 3~minutes.\\

After 1 minute, the only new squares that the beavers can reach are the squares adjacent to their starting locations. These squares are indicated in the first diagram. Notice that both beavers can reach a bike after 1 minute.\\

To find the new squares that can be reached after 2~minutes, we consider the squares that can be reached after 1 minute and determine what squares can be reached from these given 1 additional minute of travel. These new squares are indicated in the second diagram. If these squares can be reached while on a bike, then a bike symbol is indicated on the square.\\

This process is repeated to find the new squares that can be reached after 3 minutes. These squares are shown in the third diagram.\\

 Notice that the beavers have no squares in common in this third and final diagram. This means they cannot possibly meet after only 3 minutes.\\
 
Note that this final diagram can also be used to find the 4~minute route shown earlier.

\end{minipage}\hspace{10mm}
\begin{minipage}{0.43\textwidth}
\textbf{After 1 minute}\phantom{g} \\
\includegraphics[width=0.74\textwidth]{images/2021-LT-01-answ-01.png}\\
\textbf{After 2 minutes}\phantom{g} \\
\includegraphics[width=0.74\textwidth]{images/2021-LT-01-answ-02.png}\\
\textbf{After 3 minutes}\phantom{g} \\
\includegraphics[width=0.74\textwidth]{images/2021-LT-01-answ-03.png}

\end{minipage}

}

\bccBox{Connections to Computer Science}{

	In order to solve this task, we can use a technique called
	{\em breadth first search}, and specifically, a variant called
	the {\em flood fill} algorithm. \\

	Breadth first search is an {\em algorithm} or procedure that 
	searches by looking at all immediate ``next locations'' based
	on the ``current location''.  In this task, the next locations
	are the squares which are horizontally or vertically adjacent
	to the current location.  This process repeats, with the 
	most recently added ``next locations'' becoming the 
	``current location''.  When observing the pictures below, 
	the entire area is increasingly filled as if it were being flooded,
	and thus, the algorithm is known as {\em flood fill}.

\begin{center}
	\begin{tabular}{cccc}
		After one minute & After two minutes & After three minutes & 
		After four minutes \\
		\includegraphics[scale=0.2]{images/2021-LT-01-answ-01.png} &  
		\includegraphics[scale=0.2]{images/2021-LT-01-answ-02.png} & 
		\includegraphics[scale=0.2]{images/2021-LT-01-answ-03.png} &  
\includegraphics[scale=0.2]{images/2021-LT-01-answ-04.png} \\
	\end{tabular}
\end{center}

One very common use of breadth-first search is in finding the 
{\em shortest path}
between two given points, such as when finding the best sequence of directions
to drive from one location to another.  \\

Flood fill algorithms can be found
in many graphical drawing programs which contain a ``bucket fill'' tool
that will shade an entire region with a certain colour:  the algorithm will
colour adjacent {\em pixels} the same colour until it locates a boundary
pixel of a different colour.


}

\bccBox{Country of Original Author}{ 
Lithuania \hfill \raisebox{5ex-\height}{\includegraphics[width=0.12\textwidth]{images/Flag_of_Lithuania.png}}
}
\end{solution}
